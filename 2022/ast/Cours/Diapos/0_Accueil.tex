\PassOptionsToPackage{unicode=true}{hyperref} % options for packages loaded elsewhere
\PassOptionsToPackage{hyphens}{url}
%
\documentclass[10pt,xcolor=table,color={dvipsnames,usenames},ignorenonframetext,usepdftitle=false,french]{beamer}
\setbeamertemplate{caption}[numbered]
\setbeamertemplate{caption label separator}{: }
\setbeamercolor{caption name}{fg=normal text.fg}
\beamertemplatenavigationsymbolsempty
\usepackage{caption}
\captionsetup{skip=0pt,belowskip=0pt}
%\setlength\abovecaptionskip{-15pt}
\usepackage{lmodern}
\usepackage{amssymb,amsmath,mathtools,multirow}
\usepackage{float,hhline}
\usepackage{tikz}
\usepackage[tikz]{bclogo}
\usepackage{fontawesome5}
\usepackage{ifxetex,ifluatex}
\usepackage{fixltx2e} % provides \textsubscript
\ifnum 0\ifxetex 1\fi\ifluatex 1\fi=0 % if pdftex
  \usepackage[T1]{fontenc}
  \usepackage[utf8]{inputenc}
  \usepackage{textcomp} % provides euro and other symbols
\else % if luatex or xelatex
  \usepackage{unicode-math}
  \defaultfontfeatures{Ligatures=TeX,Scale=MatchLowercase}
\fi
\usetheme[coding=utf8,language=french,
,titlepagelogo=img/logo
]{TorinoTh}
% use upquote if available, for straight quotes in verbatim environments
\IfFileExists{upquote.sty}{\usepackage{upquote}}{}
% use microtype if available
\IfFileExists{microtype.sty}{%
\usepackage[]{microtype}
\UseMicrotypeSet[protrusion]{basicmath} % disable protrusion for tt fonts
}{}
\IfFileExists{parskip.sty}{%
\usepackage{parskip}
}{% else
\setlength{\parindent}{0pt}
\setlength{\parskip}{6pt plus 2pt minus 1pt}
}
\usepackage{hyperref}
\hypersetup{
            pdftitle={0 - Accueil des stagiaires},
            pdfauthor={Alain Quartier-la-Tente},
            pdfborder={0 0 0},
            breaklinks=true}
\urlstyle{same}  % don't use monospace font for urls
\newif\ifbibliography
% Prevent slide breaks in the middle of a paragraph:
\widowpenalties 1 10000
\raggedbottom
\AtBeginPart{
  \let\insertpartnumber\relax
  \let\partname\relax
  \frame{\partpage}
}
\AtBeginSection{
  \ifbibliography
  \else
    \begin{frame}{Sommaire}
    \tableofcontents[currentsection, hideothersubsections]
    \end{frame}
  \fi
}
\setlength{\emergencystretch}{3em}  % prevent overfull lines
\providecommand{\tightlist}{%
  %\setlength{\itemsep}{0pt}
  \setlength{\parskip}{0pt}
  }
\setcounter{secnumdepth}{0}

% set default figure placement to htbp
\makeatletter
\def\fps@figure{htbp}
\makeatother


\title{0 - Accueil des stagiaires}
\ateneo{Analyse des séries temporelles avec R}
\author{Alain Quartier-la-Tente}
\date{}


\setrellabel{}

\setcandidatelabel{}

\rel{}
\division{Insee}

\departement{}
\makeatletter
\let\@@magyar@captionfix\relax
\makeatother

\DeclareMathOperator{\Cov}{Cov}
\newcommand{\E}[1]{\mathbb{E}\left[ #1 \right]}
\newcommand{\V}[1]{\mathbb{V}\left[ #1 \right]}
\newcommand{\cov}[2]{\Cov\left( #1\,,\,#2 \right)}

\begin{document}
\frame[plain,noframenumbering]{\titlepage}

\begin{frame}{Faire connaissance}
\protect\hypertarget{faire-connaissance}{}
Un tour de table pour faire connaissance :

Prénom, Nom\\
Service ou établissement d'origine\\
Fonction occupée\\
Expérience sur \faIcon{r-project} et sur les séries temporelles

Pourquoi suivez-vous cette formation ? Qu'attendez vous de cette
formation ?\\
À quelles questions souhaitez-vous obtenir une réponse ?
\end{frame}

\begin{frame}[fragile]{Objectifs de la formation}
\protect\hypertarget{objectifs-de-la-formation}{}
Ensemble des documents disponibles sous :
\url{https://aqlt-formation-ast.netlify.app}

\begin{itemize}
\item
  Utilisation de \faIcon{r-project} et RStudio
\item
  Manipulation des séries temporelles sous \faIcon{r-project} :
  \texttt{ts} et \texttt{tsibble} principalement ; \texttt{xts},
  \texttt{zoo} pour quelques manipulations
\item
  Analyses graphiques d'une série temporelle : chronogramme, lag plot
  (diagramme retardé), autocorrélogrammes, saisonnalité, analyse
  spectrale
\item
  Décomposition d'une série temporelle
\item
  Prévision sans régresseur externe avec :

  \begin{itemize}
  \item
    Le lissage exponentiel
  \item
    La stationnarité et la modélisation ARIMA
  \end{itemize}
\end{itemize}

\pause

\bcinfo À approfondir : modèles de régression et de prévision,
désaisonnalisation et correction des jours ouvrables, analyse de données
à haute fréquence
\end{frame}

\begin{frame}[fragile]{Bibliographie}
\protect\hypertarget{bibliographie}{}
Aragon, Y. (2011), \emph{Séries temporelles avec R. Méthodes et cas},
Springer.

Brockwell, P. J. and Davis, R. A. (1991) \emph{Time Series: Theory and
Methods}. Second edition. Springer.

Avec \texttt{ts()} :

Hyndman, R.J., \& Athanasopoulos, G. (2018) \emph{Forecasting:
principles and practice}, 2nd edition, OTexts: Melbourne, Australia.
OTexts.com/fpp2. Accessed on nov. 2022.

Sur \texttt{tsibble} :

Hyndman, R.J., \& Athanasopoulos, G. (2021) \emph{Forecasting:
principles and practice}, 3rd edition, OTexts: Melbourne, Australia.
OTexts.com/fpp3. Accessed on nov. 2022.
\end{frame}

\end{document}
