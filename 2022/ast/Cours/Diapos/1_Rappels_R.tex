\PassOptionsToPackage{unicode=true}{hyperref} % options for packages loaded elsewhere
\PassOptionsToPackage{hyphens}{url}
%
\documentclass[10pt,xcolor=table,color={dvipsnames,usenames},ignorenonframetext,usepdftitle=false,french]{beamer}
\setbeamertemplate{caption}[numbered]
\setbeamertemplate{caption label separator}{: }
\setbeamercolor{caption name}{fg=normal text.fg}
\beamertemplatenavigationsymbolsempty
\usepackage{caption}
\captionsetup{skip=0pt,belowskip=0pt}
%\setlength\abovecaptionskip{-15pt}
\usepackage{lmodern}
\usepackage{amssymb,amsmath,mathtools,multirow}
\usepackage{float,hhline}
\usepackage{tikz}
\usepackage[tikz]{bclogo}
\usepackage{fontawesome5}
\usepackage{ifxetex,ifluatex}
\usepackage{fixltx2e} % provides \textsubscript
\ifnum 0\ifxetex 1\fi\ifluatex 1\fi=0 % if pdftex
  \usepackage[T1]{fontenc}
  \usepackage[utf8]{inputenc}
  \usepackage{textcomp} % provides euro and other symbols
\else % if luatex or xelatex
  \usepackage{unicode-math}
  \defaultfontfeatures{Ligatures=TeX,Scale=MatchLowercase}
\fi
\usetheme[coding=utf8,language=french,
,titlepagelogo=img/logo
]{TorinoTh}
% use upquote if available, for straight quotes in verbatim environments
\IfFileExists{upquote.sty}{\usepackage{upquote}}{}
% use microtype if available
\IfFileExists{microtype.sty}{%
\usepackage[]{microtype}
\UseMicrotypeSet[protrusion]{basicmath} % disable protrusion for tt fonts
}{}
\IfFileExists{parskip.sty}{%
\usepackage{parskip}
}{% else
\setlength{\parindent}{0pt}
\setlength{\parskip}{6pt plus 2pt minus 1pt}
}
\usepackage{hyperref}
\hypersetup{
            pdftitle={1 - Rappels sur l'environnement de travail de R},
            pdfauthor={Alain Quartier-la-Tente},
            pdfborder={0 0 0},
            breaklinks=true}
\urlstyle{same}  % don't use monospace font for urls
\newif\ifbibliography
\usepackage{color}
\usepackage{fancyvrb}
\newcommand{\VerbBar}{|}
\newcommand{\VERB}{\Verb[commandchars=\\\{\}]}
\DefineVerbatimEnvironment{Highlighting}{Verbatim}{commandchars=\\\{\}}
% Add ',fontsize=\small' for more characters per line
\usepackage{framed}
\definecolor{shadecolor}{RGB}{248,248,248}
\newenvironment{Shaded}{\begin{snugshade}}{\end{snugshade}}
\newcommand{\AlertTok}[1]{\textcolor[rgb]{0.94,0.16,0.16}{#1}}
\newcommand{\AnnotationTok}[1]{\textcolor[rgb]{0.56,0.35,0.01}{\textbf{\textit{#1}}}}
\newcommand{\AttributeTok}[1]{\textcolor[rgb]{0.77,0.63,0.00}{#1}}
\newcommand{\BaseNTok}[1]{\textcolor[rgb]{0.00,0.00,0.81}{#1}}
\newcommand{\BuiltInTok}[1]{#1}
\newcommand{\CharTok}[1]{\textcolor[rgb]{0.31,0.60,0.02}{#1}}
\newcommand{\CommentTok}[1]{\textcolor[rgb]{0.56,0.35,0.01}{\textit{#1}}}
\newcommand{\CommentVarTok}[1]{\textcolor[rgb]{0.56,0.35,0.01}{\textbf{\textit{#1}}}}
\newcommand{\ConstantTok}[1]{\textcolor[rgb]{0.00,0.00,0.00}{#1}}
\newcommand{\ControlFlowTok}[1]{\textcolor[rgb]{0.13,0.29,0.53}{\textbf{#1}}}
\newcommand{\DataTypeTok}[1]{\textcolor[rgb]{0.13,0.29,0.53}{#1}}
\newcommand{\DecValTok}[1]{\textcolor[rgb]{0.00,0.00,0.81}{#1}}
\newcommand{\DocumentationTok}[1]{\textcolor[rgb]{0.56,0.35,0.01}{\textbf{\textit{#1}}}}
\newcommand{\ErrorTok}[1]{\textcolor[rgb]{0.64,0.00,0.00}{\textbf{#1}}}
\newcommand{\ExtensionTok}[1]{#1}
\newcommand{\FloatTok}[1]{\textcolor[rgb]{0.00,0.00,0.81}{#1}}
\newcommand{\FunctionTok}[1]{\textcolor[rgb]{0.00,0.00,0.00}{#1}}
\newcommand{\ImportTok}[1]{#1}
\newcommand{\InformationTok}[1]{\textcolor[rgb]{0.56,0.35,0.01}{\textbf{\textit{#1}}}}
\newcommand{\KeywordTok}[1]{\textcolor[rgb]{0.13,0.29,0.53}{\textbf{#1}}}
\newcommand{\NormalTok}[1]{#1}
\newcommand{\OperatorTok}[1]{\textcolor[rgb]{0.81,0.36,0.00}{\textbf{#1}}}
\newcommand{\OtherTok}[1]{\textcolor[rgb]{0.56,0.35,0.01}{#1}}
\newcommand{\PreprocessorTok}[1]{\textcolor[rgb]{0.56,0.35,0.01}{\textit{#1}}}
\newcommand{\RegionMarkerTok}[1]{#1}
\newcommand{\SpecialCharTok}[1]{\textcolor[rgb]{0.00,0.00,0.00}{#1}}
\newcommand{\SpecialStringTok}[1]{\textcolor[rgb]{0.31,0.60,0.02}{#1}}
\newcommand{\StringTok}[1]{\textcolor[rgb]{0.31,0.60,0.02}{#1}}
\newcommand{\VariableTok}[1]{\textcolor[rgb]{0.00,0.00,0.00}{#1}}
\newcommand{\VerbatimStringTok}[1]{\textcolor[rgb]{0.31,0.60,0.02}{#1}}
\newcommand{\WarningTok}[1]{\textcolor[rgb]{0.56,0.35,0.01}{\textbf{\textit{#1}}}}
% Prevent slide breaks in the middle of a paragraph:
\widowpenalties 1 10000
\raggedbottom
\AtBeginPart{
  \let\insertpartnumber\relax
  \let\partname\relax
  \frame{\partpage}
}
\AtBeginSection{
  \ifbibliography
  \else
    \begin{frame}{Sommaire}
    \tableofcontents[currentsection, hideothersubsections]
    \end{frame}
  \fi
}
\setlength{\emergencystretch}{3em}  % prevent overfull lines
\providecommand{\tightlist}{%
  %\setlength{\itemsep}{0pt}
  \setlength{\parskip}{0pt}
  }
\setcounter{secnumdepth}{0}

% set default figure placement to htbp
\makeatletter
\def\fps@figure{htbp}
\makeatother


\title{1 - Rappels sur l'environnement de travail de R}
\ateneo{Analyse des séries temporelles avec R}
\author{Alain Quartier-la-Tente}
\date{}


\setrellabel{}

\setcandidatelabel{}

\rel{}
\division{Insee}

\departement{}
\makeatletter
\let\@@magyar@captionfix\relax
\makeatother

\DeclareMathOperator{\Cov}{Cov}
\newcommand{\E}[1]{\mathbb{E}\left[ #1 \right]}
\newcommand{\V}[1]{\mathbb{V}\left[ #1 \right]}
\newcommand{\cov}[2]{\Cov\left( #1\,,\,#2 \right)}

\begin{document}
\frame[plain,noframenumbering]{\titlepage}

\begin{frame}{\faIcon{r-project}}
\protect\hypertarget{section}{}
\faIcon{r-project} : Logiciel statistique

RStudio : IDE le plus connu

CRAN : Plateforme centralisant un ensemble de packages R sous un format
normalisé permettant une installation facile

GitHub : Plateforme de partage de code où l'on retrouve de nombreux
packages en développement
\end{frame}

\begin{frame}[fragile]{Aide}
\protect\hypertarget{aide}{}
\begin{itemize}
\item
  Si vous ne connaissez pas les fonctions, Google est votre ami
\item
  Sinon \texttt{help(ma\_fonction)} ou \texttt{?ma\_fonction} pour
  chercher l'aide associée à \texttt{ma\_fonction}. Voir aussi vignettes
  (documentation long format). Exemple :
\end{itemize}

\begin{Shaded}
\begin{Highlighting}[]
\CommentTok{\# Pour voir l\textquotesingle{}ensemble des vignettes du package grid}
\FunctionTok{vignette}\NormalTok{(}\AttributeTok{package =} \StringTok{"grid"}\NormalTok{)}
\CommentTok{\# Pour afficher une vignette en utilisant son nom}
\FunctionTok{vignette}\NormalTok{(}\StringTok{"moveline"}\NormalTok{, }\AttributeTok{package =} \StringTok{"grid"}\NormalTok{)}
\end{Highlighting}
\end{Shaded}

\begin{itemize}
\tightlist
\item
  Cran Task Views (\url{https://cran.r-project.org/web/views/})
  regroupement de packages selon des thèmes particuliers. Exemple pour
  ce cours : \url{https://cran.r-project.org/web/views/TimeSeries.html}
\end{itemize}
\end{frame}

\hypertarget{les-types-de-base}{%
\section{Les types de base}\label{les-types-de-base}}

\hypertarget{les-vecteurs}{%
\subsection{Les vecteurs}\label{les-vecteurs}}

\begin{frame}[fragile,allowframebreaks]{Les vecteurs}
\protect\hypertarget{les-vecteurs-1}{}
Les vecteurs sont les objets les plus simples : créés avec fonction
\texttt{c()} et leurs éléments peuvent être manipulés avec l'opérateur
\texttt{{[}}

\begin{Shaded}
\begin{Highlighting}[]
\NormalTok{v1 }\OtherTok{\textless{}{-}} \FunctionTok{c}\NormalTok{(}\DecValTok{1}\NormalTok{, }\DecValTok{2}\NormalTok{, }\DecValTok{3}\NormalTok{); v2 }\OtherTok{\textless{}{-}} \FunctionTok{c}\NormalTok{(}\StringTok{"a"}\NormalTok{, }\StringTok{"b"}\NormalTok{)}
\NormalTok{v1}
\end{Highlighting}
\end{Shaded}

\begin{verbatim}
## [1] 1 2 3
\end{verbatim}

\begin{Shaded}
\begin{Highlighting}[]
\NormalTok{v2}
\end{Highlighting}
\end{Shaded}

\begin{verbatim}
## [1] "a" "b"
\end{verbatim}

\begin{Shaded}
\begin{Highlighting}[]
\CommentTok{\# v1 peut aussi se créer de façon équivalente avec :}
\DecValTok{1}\SpecialCharTok{:}\DecValTok{3}
\end{Highlighting}
\end{Shaded}

\begin{verbatim}
## [1] 1 2 3
\end{verbatim}

\begin{Shaded}
\begin{Highlighting}[]
\CommentTok{\# Pour concaténer deux vecteurs, notez le changement de type}
\NormalTok{v3 }\OtherTok{\textless{}{-}} \FunctionTok{c}\NormalTok{(v1, v2)}
\NormalTok{v3}
\end{Highlighting}
\end{Shaded}

\begin{verbatim}
## [1] "1" "2" "3" "a" "b"
\end{verbatim}

\begin{Shaded}
\begin{Highlighting}[]
\NormalTok{v3[}\FunctionTok{c}\NormalTok{(}\DecValTok{4}\NormalTok{, }\DecValTok{1}\NormalTok{)] }\CommentTok{\# 4e puis 1er élément}
\end{Highlighting}
\end{Shaded}

\begin{verbatim}
## [1] "a" "1"
\end{verbatim}

\begin{Shaded}
\begin{Highlighting}[]
\NormalTok{v3[}\SpecialCharTok{{-}}\FunctionTok{c}\NormalTok{(}\DecValTok{4}\NormalTok{, }\DecValTok{1}\NormalTok{)] }\CommentTok{\# on enlève 1er et 4e éléments}
\end{Highlighting}
\end{Shaded}

\begin{verbatim}
## [1] "2" "3" "b"
\end{verbatim}

\begin{Shaded}
\begin{Highlighting}[]
\CommentTok{\# Les éléments peuvent également être nommés}
\NormalTok{v4 }\OtherTok{\textless{}{-}} \FunctionTok{c}\NormalTok{(}\AttributeTok{elem1 =} \DecValTok{1}\NormalTok{, }\AttributeTok{elem2 =} \DecValTok{2}\NormalTok{, }\DecValTok{4}\NormalTok{)}
\NormalTok{v4}
\end{Highlighting}
\end{Shaded}

\begin{verbatim}
## elem1 elem2       
##     1     2     4
\end{verbatim}

\begin{Shaded}
\begin{Highlighting}[]
\FunctionTok{names}\NormalTok{(v4)}
\end{Highlighting}
\end{Shaded}

\begin{verbatim}
## [1] "elem1" "elem2" ""
\end{verbatim}

\begin{Shaded}
\begin{Highlighting}[]
\FunctionTok{names}\NormalTok{(v4)[}\DecValTok{1}\NormalTok{] }\OtherTok{\textless{}{-}} \StringTok{"toto"}
\NormalTok{v4}
\end{Highlighting}
\end{Shaded}

\begin{verbatim}
##  toto elem2       
##     1     2     4
\end{verbatim}

\begin{Shaded}
\begin{Highlighting}[]
\NormalTok{v4[}\FunctionTok{c}\NormalTok{(}\StringTok{"toto"}\NormalTok{, }\StringTok{"elem2"}\NormalTok{)]}
\end{Highlighting}
\end{Shaded}

\begin{verbatim}
##  toto elem2 
##     1     2
\end{verbatim}
\end{frame}

\hypertarget{les-matrices}{%
\subsection{Les matrices}\label{les-matrices}}

\begin{frame}[fragile,allowframebreaks]{Les matrices}
\protect\hypertarget{les-matrices-1}{}
Matrices : vecteurs à deux dimensions créés avec fonction
\texttt{matrix()}

\begin{Shaded}
\begin{Highlighting}[]
\NormalTok{m1 }\OtherTok{\textless{}{-}} \FunctionTok{matrix}\NormalTok{(}\DecValTok{1}\SpecialCharTok{:}\DecValTok{12}\NormalTok{, }\AttributeTok{ncol =} \DecValTok{3}\NormalTok{); m2 }\OtherTok{\textless{}{-}} \FunctionTok{matrix}\NormalTok{(}\DecValTok{1}\SpecialCharTok{:}\DecValTok{12}\NormalTok{, }\AttributeTok{nrow =} \DecValTok{3}\NormalTok{)}
\NormalTok{m1; }\FunctionTok{t}\NormalTok{(m1); m1 }\SpecialCharTok{*} \DecValTok{2}
\end{Highlighting}
\end{Shaded}

\begin{verbatim}
##      [,1] [,2] [,3]
## [1,]    1    5    9
## [2,]    2    6   10
## [3,]    3    7   11
## [4,]    4    8   12
\end{verbatim}

\begin{verbatim}
##      [,1] [,2] [,3] [,4]
## [1,]    1    2    3    4
## [2,]    5    6    7    8
## [3,]    9   10   11   12
\end{verbatim}

\begin{verbatim}
##      [,1] [,2] [,3]
## [1,]    2   10   18
## [2,]    4   12   20
## [3,]    6   14   22
## [4,]    8   16   24
\end{verbatim}

\begin{Shaded}
\begin{Highlighting}[]
\NormalTok{m1 }\SpecialCharTok{\%*\%}\NormalTok{ m2 }\CommentTok{\# multiplication matricielle}
\end{Highlighting}
\end{Shaded}

\begin{verbatim}
##      [,1] [,2] [,3] [,4]
## [1,]   38   83  128  173
## [2,]   44   98  152  206
## [3,]   50  113  176  239
## [4,]   56  128  200  272
\end{verbatim}

\begin{Shaded}
\begin{Highlighting}[]
\NormalTok{m1[, }\DecValTok{1}\NormalTok{] }\CommentTok{\# 1ere colonne : c\textquotesingle{}est un vecteur}
\end{Highlighting}
\end{Shaded}

\begin{verbatim}
## [1] 1 2 3 4
\end{verbatim}

\begin{Shaded}
\begin{Highlighting}[]
\NormalTok{m1[}\SpecialCharTok{{-}}\DecValTok{2}\NormalTok{, ] }\CommentTok{\# Tout sauf 2ème ligne}
\end{Highlighting}
\end{Shaded}

\begin{verbatim}
##      [,1] [,2] [,3]
## [1,]    1    5    9
## [2,]    3    7   11
## [3,]    4    8   12
\end{verbatim}

\begin{Shaded}
\begin{Highlighting}[]
\CommentTok{\# Nombre de lignes et de colonnes :}
\FunctionTok{nrow}\NormalTok{(m1); }\FunctionTok{ncol}\NormalTok{(m1); }\FunctionTok{dim}\NormalTok{(m1)}
\end{Highlighting}
\end{Shaded}

\begin{verbatim}
## [1] 4
\end{verbatim}

\begin{verbatim}
## [1] 3
\end{verbatim}

\begin{verbatim}
## [1] 4 3
\end{verbatim}

\begin{Shaded}
\begin{Highlighting}[]
\CommentTok{\# De la même façon que pour les vecteurs on peut nommer lignes/colonnes}
\FunctionTok{colnames}\NormalTok{(m1) }\OtherTok{\textless{}{-}} \FunctionTok{paste0}\NormalTok{(}\StringTok{"col"}\NormalTok{, }\DecValTok{1}\SpecialCharTok{:}\FunctionTok{ncol}\NormalTok{(m1))}
\FunctionTok{rownames}\NormalTok{(m1) }\OtherTok{\textless{}{-}} \FunctionTok{paste0}\NormalTok{(}\StringTok{"row"}\NormalTok{, }\DecValTok{1}\SpecialCharTok{:}\FunctionTok{nrow}\NormalTok{(m1))}
\NormalTok{m1}
\end{Highlighting}
\end{Shaded}

\begin{verbatim}
##      col1 col2 col3
## row1    1    5    9
## row2    2    6   10
## row3    3    7   11
## row4    4    8   12
\end{verbatim}

\begin{Shaded}
\begin{Highlighting}[]
\NormalTok{m1[, }\StringTok{"col2"}\NormalTok{]}
\end{Highlighting}
\end{Shaded}

\begin{verbatim}
## row1 row2 row3 row4 
##    5    6    7    8
\end{verbatim}

\begin{Shaded}
\begin{Highlighting}[]
\CommentTok{\# Pour combiner des matrices, on peut utiliser cbind et rbind:}
\FunctionTok{cbind}\NormalTok{(m1, }\DecValTok{1}\SpecialCharTok{:}\DecValTok{4}\NormalTok{)}
\end{Highlighting}
\end{Shaded}

\begin{verbatim}
##      col1 col2 col3  
## row1    1    5    9 1
## row2    2    6   10 2
## row3    3    7   11 3
## row4    4    8   12 4
\end{verbatim}

\begin{Shaded}
\begin{Highlighting}[]
\FunctionTok{rbind}\NormalTok{(m1, m1)}
\end{Highlighting}
\end{Shaded}

\begin{verbatim}
##      col1 col2 col3
## row1    1    5    9
## row2    2    6   10
## row3    3    7   11
## row4    4    8   12
## row1    1    5    9
## row2    2    6   10
## row3    3    7   11
## row4    4    8   12
\end{verbatim}

On peut utiliser la fonction \texttt{apply} pour appliquer une fonction
à toutes les lignes ou toutes les colonnes. Exemple :

\begin{Shaded}
\begin{Highlighting}[]
\FunctionTok{apply}\NormalTok{(m1, }\DecValTok{1}\NormalTok{, sum) }\CommentTok{\# somme sur toutes les lignes (dimension 1)}
\end{Highlighting}
\end{Shaded}

\begin{verbatim}
## row1 row2 row3 row4 
##   15   18   21   24
\end{verbatim}

\begin{Shaded}
\begin{Highlighting}[]
\FunctionTok{apply}\NormalTok{(m1, }\DecValTok{2}\NormalTok{, sum) }\CommentTok{\# somme sur toutes les colonnes (dimension 2)}
\end{Highlighting}
\end{Shaded}

\begin{verbatim}
## col1 col2 col3 
##   10   26   42
\end{verbatim}
\end{frame}

\hypertarget{les-listes}{%
\subsection{Les listes}\label{les-listes}}

\begin{frame}[fragile,allowframebreaks]{Les listes}
\protect\hypertarget{les-listes-1}{}
Une liste peut contenir tout type d'objet

\begin{Shaded}
\begin{Highlighting}[]
\NormalTok{l1 }\OtherTok{\textless{}{-}} \FunctionTok{list}\NormalTok{(v1, m1, v4); l1}
\end{Highlighting}
\end{Shaded}

\begin{verbatim}
## [[1]]
## [1] 1 2 3
## 
## [[2]]
##      col1 col2 col3
## row1    1    5    9
## row2    2    6   10
## row3    3    7   11
## row4    4    8   12
## 
## [[3]]
##  toto elem2       
##     1     2     4
\end{verbatim}

\begin{Shaded}
\begin{Highlighting}[]
\FunctionTok{length}\NormalTok{(l1) }\CommentTok{\# nombre d\textquotesingle{}éléments d\textquotesingle{}une liste}
\end{Highlighting}
\end{Shaded}

\begin{verbatim}
## [1] 3
\end{verbatim}

\begin{Shaded}
\begin{Highlighting}[]
\CommentTok{\# On peut encore nommer les éléments de la liste :}
\FunctionTok{names}\NormalTok{(l1) }\OtherTok{\textless{}{-}} \FunctionTok{c}\NormalTok{(}\StringTok{"vect1"}\NormalTok{, }\StringTok{"mat"}\NormalTok{, }\StringTok{"vect2"}\NormalTok{)}
\NormalTok{l1}
\end{Highlighting}
\end{Shaded}

\begin{verbatim}
## $vect1
## [1] 1 2 3
## 
## $mat
##      col1 col2 col3
## row1    1    5    9
## row2    2    6   10
## row3    3    7   11
## row4    4    8   12
## 
## $vect2
##  toto elem2       
##     1     2     4
\end{verbatim}

\begin{Shaded}
\begin{Highlighting}[]
\CommentTok{\# Pour accéder à un élément d\textquotesingle{}une liste utiliser [[,}
\CommentTok{\# autrement on a encore une liste}
\NormalTok{l1[}\DecValTok{1}\NormalTok{] }\CommentTok{\# liste d\textquotesingle{}un seul élément : v1}
\end{Highlighting}
\end{Shaded}

\begin{verbatim}
## $vect1
## [1] 1 2 3
\end{verbatim}

\begin{Shaded}
\begin{Highlighting}[]
\NormalTok{l1[[}\DecValTok{1}\NormalTok{]] }\CommentTok{\# premier élément de la liste}
\end{Highlighting}
\end{Shaded}

\begin{verbatim}
## [1] 1 2 3
\end{verbatim}

\begin{Shaded}
\begin{Highlighting}[]
\CommentTok{\# On concatène deux listes avec fonction c:}
\FunctionTok{c}\NormalTok{(l1, l1[}\SpecialCharTok{{-}}\DecValTok{2}\NormalTok{])}
\end{Highlighting}
\end{Shaded}

\begin{verbatim}
## $vect1
## [1] 1 2 3
## 
## $mat
##      col1 col2 col3
## row1    1    5    9
## row2    2    6   10
## row3    3    7   11
## row4    4    8   12
## 
## $vect2
##  toto elem2       
##     1     2     4 
## 
## $vect1
## [1] 1 2 3
## 
## $vect2
##  toto elem2       
##     1     2     4
\end{verbatim}
\end{frame}

\hypertarget{le-data.frame-et-tibble}{%
\subsection{Le data.frame et tibble}\label{le-data.frame-et-tibble}}

\begin{frame}[fragile,allowframebreaks]{Le data.frame}
\protect\hypertarget{le-data.frame}{}
Entre les listes et matrices : comme un tableur, souvent utilisé pour
stocker des données

\begin{Shaded}
\begin{Highlighting}[]
\NormalTok{d1 }\OtherTok{\textless{}{-}} \FunctionTok{data.frame}\NormalTok{(}\AttributeTok{col1 =} \FunctionTok{c}\NormalTok{(}\StringTok{"a"}\NormalTok{, }\StringTok{"b"}\NormalTok{, }\StringTok{"c"}\NormalTok{), }\AttributeTok{col2 =} \DecValTok{1}\SpecialCharTok{:}\DecValTok{3}\NormalTok{)}
\NormalTok{d1}
\end{Highlighting}
\end{Shaded}

\begin{verbatim}
##   col1 col2
## 1    a    1
## 2    b    2
## 3    c    3
\end{verbatim}
\end{frame}

\begin{frame}[fragile,allowframebreaks]{Le tibble}
\protect\hypertarget{le-tibble}{}
\texttt{tibble} : comme un data.frame réinventé, plus permissif

\begin{Shaded}
\begin{Highlighting}[]
\FunctionTok{library}\NormalTok{(tibble)}
\NormalTok{t1 }\OtherTok{\textless{}{-}} \FunctionTok{tibble}\NormalTok{(}\AttributeTok{col1 =} \FunctionTok{c}\NormalTok{(}\StringTok{"a"}\NormalTok{, }\StringTok{"b"}\NormalTok{, }\StringTok{"c"}\NormalTok{), }\AttributeTok{col2 =} \DecValTok{1}\SpecialCharTok{:}\DecValTok{3}\NormalTok{)}
\NormalTok{t1 }\CommentTok{\# ou as.tibble(d1)}
\end{Highlighting}
\end{Shaded}

\begin{verbatim}
## # A tibble: 3 x 2
##   col1   col2
##   <chr> <int>
## 1 a         1
## 2 b         2
## 3 c         3
\end{verbatim}

\begin{Shaded}
\begin{Highlighting}[]
\CommentTok{\# On peut aussi les définir ligne par ligne :}
\FunctionTok{tribble}\NormalTok{(}
  \SpecialCharTok{\textasciitilde{}}\NormalTok{col1, }\SpecialCharTok{\textasciitilde{}}\NormalTok{col2,}
  \StringTok{"a"}\NormalTok{, }\DecValTok{1}\NormalTok{,}
  \StringTok{"b"}\NormalTok{, }\DecValTok{2}\NormalTok{,}
  \StringTok{"c"}\NormalTok{, }\DecValTok{3}
\NormalTok{)}
\end{Highlighting}
\end{Shaded}

\begin{verbatim}
## # A tibble: 3 x 2
##   col1   col2
##   <chr> <dbl>
## 1 a         1
## 2 b         2
## 3 c         3
\end{verbatim}
\end{frame}

\hypertarget{importation-des-donnuxe9es}{%
\section{Importation des données}\label{importation-des-donnuxe9es}}

\begin{frame}{Importer des données}
\protect\hypertarget{importer-des-donnuxe9es}{}
Soyez fainéants et commencez par utiliser l'interface de RStudio
(Environnement \textgreater{} Import Dataset).
\end{frame}

\hypertarget{les-suxe9ries-temporelles}{%
\section{Les séries temporelles}\label{les-suxe9ries-temporelles}}

\begin{frame}[fragile]{\texttt{ts()}}
\protect\hypertarget{ts}{}
Il existe de nombreux formats pour gérer les séries temporelles. Dans
cette formation nous verrons :

\begin{itemize}
\item
  \texttt{ts()} : format de base R simple à utiliser mais des
  difficultés à gérer les fréquences non-entières (journalières,
  hebdomadaires, etc.)
\item
  \texttt{tsibble()} : inspiré du \texttt{tidyverse} (\texttt{tidyverts}
  \url{https://tidyverts.org}) mais pour la gestion des séries
  temporelles
\end{itemize}
\end{frame}

\begin{frame}[fragile,allowframebreaks]{\texttt{ts()}}
\protect\hypertarget{ts-1}{}
On peut créer un objet avec la fonction
\texttt{ts(data\ =\ .,\ start\ =\ .,\ frequency\ =\ .)}

\begin{Shaded}
\begin{Highlighting}[]
\NormalTok{x }\OtherTok{=} \FunctionTok{ts}\NormalTok{(}\FunctionTok{c}\NormalTok{(}\DecValTok{1}\SpecialCharTok{:}\DecValTok{12}\NormalTok{), }\AttributeTok{start =} \DecValTok{2020}\NormalTok{, }\AttributeTok{frequency =} \DecValTok{4}\NormalTok{)}
\NormalTok{x; }\FunctionTok{class}\NormalTok{(x)}
\end{Highlighting}
\end{Shaded}

\begin{verbatim}
##      Qtr1 Qtr2 Qtr3 Qtr4
## 2020    1    2    3    4
## 2021    5    6    7    8
## 2022    9   10   11   12
\end{verbatim}

\begin{verbatim}
## [1] "ts"
\end{verbatim}

\begin{Shaded}
\begin{Highlighting}[]
\NormalTok{mts }\OtherTok{\textless{}{-}} \FunctionTok{ts}\NormalTok{(}\FunctionTok{matrix}\NormalTok{(}\FunctionTok{rnorm}\NormalTok{(}\DecValTok{30}\NormalTok{), }\DecValTok{10}\NormalTok{, }\DecValTok{3}\NormalTok{), }\AttributeTok{start =} \FunctionTok{c}\NormalTok{(}\DecValTok{1961}\NormalTok{, }\DecValTok{1}\NormalTok{),}
          \AttributeTok{frequency =} \DecValTok{12}\NormalTok{)}
\NormalTok{mts; }\FunctionTok{class}\NormalTok{(mts)}
\end{Highlighting}
\end{Shaded}

\begin{verbatim}
##            Series 1    Series 2    Series 3
## Jan 1961 -0.6264538  1.51178117  0.91897737
## Feb 1961  0.1836433  0.38984324  0.78213630
## Mar 1961 -0.8356286 -0.62124058  0.07456498
## Apr 1961  1.5952808 -2.21469989 -1.98935170
## May 1961  0.3295078  1.12493092  0.61982575
## Jun 1961 -0.8204684 -0.04493361 -0.05612874
## Jul 1961  0.4874291 -0.01619026 -0.15579551
## Aug 1961  0.7383247  0.94383621 -1.47075238
## Sep 1961  0.5757814  0.82122120 -0.47815006
## Oct 1961 -0.3053884  0.59390132  0.41794156
\end{verbatim}

\begin{verbatim}
## [1] "mts"    "ts"     "matrix"
\end{verbatim}

Pour manipulations : voir TP
\end{frame}

\begin{frame}[fragile,allowframebreaks]{\texttt{tsibble}}
\protect\hypertarget{tsibble}{}
\begin{Shaded}
\begin{Highlighting}[]
\FunctionTok{library}\NormalTok{(tsibble)}
\NormalTok{tsibbledata}\SpecialCharTok{::}\NormalTok{aus\_production}
\end{Highlighting}
\end{Shaded}

\begin{verbatim}
## # A tsibble: 218 x 7 [1Q]
##    Quarter  Beer Tobacco Bricks Cement Electricity   Gas
##      <qtr> <dbl>   <dbl>  <dbl>  <dbl>       <dbl> <dbl>
##  1 1956 Q1   284    5225    189    465        3923     5
##  2 1956 Q2   213    5178    204    532        4436     6
##  3 1956 Q3   227    5297    208    561        4806     7
##  4 1956 Q4   308    5681    197    570        4418     6
##  5 1957 Q1   262    5577    187    529        4339     5
##  6 1957 Q2   228    5651    214    604        4811     7
##  7 1957 Q3   236    5317    227    603        5259     7
##  8 1957 Q4   320    6152    222    582        4735     6
##  9 1958 Q1   272    5758    199    554        4608     5
## 10 1958 Q2   233    5641    229    620        5196     7
## # ... with 208 more rows
\end{verbatim}

\begin{Shaded}
\begin{Highlighting}[]
\NormalTok{tsibbledata}\SpecialCharTok{::}\NormalTok{global\_economy}
\end{Highlighting}
\end{Shaded}

\begin{verbatim}
## # A tsibble: 15,150 x 9 [1Y]
## # Key:       Country [263]
##    Country     Code   Year         GDP Growth   CPI Imports Exports Population
##    <fct>       <fct> <dbl>       <dbl>  <dbl> <dbl>   <dbl>   <dbl>      <dbl>
##  1 Afghanistan AFG    1960  537777811.     NA    NA    7.02    4.13    8996351
##  2 Afghanistan AFG    1961  548888896.     NA    NA    8.10    4.45    9166764
##  3 Afghanistan AFG    1962  546666678.     NA    NA    9.35    4.88    9345868
##  4 Afghanistan AFG    1963  751111191.     NA    NA   16.9     9.17    9533954
##  5 Afghanistan AFG    1964  800000044.     NA    NA   18.1     8.89    9731361
##  6 Afghanistan AFG    1965 1006666638.     NA    NA   21.4    11.3     9938414
##  7 Afghanistan AFG    1966 1399999967.     NA    NA   18.6     8.57   10152331
##  8 Afghanistan AFG    1967 1673333418.     NA    NA   14.2     6.77   10372630
##  9 Afghanistan AFG    1968 1373333367.     NA    NA   15.2     8.90   10604346
## 10 Afghanistan AFG    1969 1408888922.     NA    NA   15.0    10.1    10854428
## # ... with 15,140 more rows
\end{verbatim}

\begin{Shaded}
\begin{Highlighting}[]
\FunctionTok{as\_tsibble}\NormalTok{(mts)}
\end{Highlighting}
\end{Shaded}

\begin{verbatim}
## # A tsibble: 30 x 3 [1M]
## # Key:       key [3]
##       index key       value
##       <mth> <chr>     <dbl>
##  1 1961 jan Series 1 -0.626
##  2 1961 fév Series 1  0.184
##  3 1961 mar Series 1 -0.836
##  4 1961 avr Series 1  1.60 
##  5 1961 mai Series 1  0.330
##  6 1961 jui Series 1 -0.820
##  7 1961 jul Series 1  0.487
##  8 1961 aoû Series 1  0.738
##  9 1961 sep Series 1  0.576
## 10 1961 oct Series 1 -0.305
## # ... with 20 more rows
\end{verbatim}

S'adapte assez bien au tidyverse : \texttt{index\_by()} remplace le
\texttt{group\_by()} mais sur les dates, \texttt{group\_by\_key()}
permet de le faire sur les clefs:

\begin{Shaded}
\begin{Highlighting}[]
\FunctionTok{library}\NormalTok{(dplyr)}
\FunctionTok{as\_tsibble}\NormalTok{(mts) }\SpecialCharTok{\%\textgreater{}\%}  
    \FunctionTok{index\_by}\NormalTok{() }\SpecialCharTok{\%\textgreater{}\%} 
    \FunctionTok{summarise}\NormalTok{(}\AttributeTok{moy =} \FunctionTok{mean}\NormalTok{(value))}
\end{Highlighting}
\end{Shaded}

\begin{verbatim}
## # A tsibble: 10 x 2 [1M]
##       index     moy
##       <mth>   <dbl>
##  1 1961 jan  0.601 
##  2 1961 fév  0.452 
##  3 1961 mar -0.461 
##  4 1961 avr -0.870 
##  5 1961 mai  0.691 
##  6 1961 jui -0.307 
##  7 1961 jul  0.105 
##  8 1961 aoû  0.0705
##  9 1961 sep  0.306 
## 10 1961 oct  0.235
\end{verbatim}

\begin{Shaded}
\begin{Highlighting}[]
\FunctionTok{as\_tsibble}\NormalTok{(mts) }\SpecialCharTok{\%\textgreater{}\%}  
    \CommentTok{\# index\_by() \%\textgreater{}\% }
    \FunctionTok{summarise}\NormalTok{(}\AttributeTok{moy =} \FunctionTok{mean}\NormalTok{(value))}
\end{Highlighting}
\end{Shaded}

\begin{verbatim}
## # A tsibble: 10 x 2 [1M]
##       index     moy
##       <mth>   <dbl>
##  1 1961 jan  0.601 
##  2 1961 fév  0.452 
##  3 1961 mar -0.461 
##  4 1961 avr -0.870 
##  5 1961 mai  0.691 
##  6 1961 jui -0.307 
##  7 1961 jul  0.105 
##  8 1961 aoû  0.0705
##  9 1961 sep  0.306 
## 10 1961 oct  0.235
\end{verbatim}

\begin{Shaded}
\begin{Highlighting}[]
\FunctionTok{as\_tsibble}\NormalTok{(mts) }\SpecialCharTok{\%\textgreater{}\%}  
    \FunctionTok{index\_by}\NormalTok{(}\AttributeTok{date =} \SpecialCharTok{\textasciitilde{}} \FunctionTok{yearquarter}\NormalTok{(.)) }\SpecialCharTok{\%\textgreater{}\%} 
    \FunctionTok{summarise}\NormalTok{(}\AttributeTok{moy =} \FunctionTok{mean}\NormalTok{(value))}
\end{Highlighting}
\end{Shaded}

\begin{verbatim}
## # A tsibble: 4 x 2 [1Q]
##      date    moy
##     <qtr>  <dbl>
## 1 1961 Q1  0.198
## 2 1961 Q2 -0.162
## 3 1961 Q3  0.161
## 4 1961 Q4  0.235
\end{verbatim}

\begin{Shaded}
\begin{Highlighting}[]
\FunctionTok{as\_tsibble}\NormalTok{(mts) }\SpecialCharTok{\%\textgreater{}\%}  
    \FunctionTok{group\_by\_key}\NormalTok{() }\SpecialCharTok{\%\textgreater{}\%} 
    \FunctionTok{summarise}\NormalTok{(}\AttributeTok{moy =} \FunctionTok{mean}\NormalTok{(value))}
\end{Highlighting}
\end{Shaded}

\begin{verbatim}
## # A tsibble: 30 x 3 [1M]
## # Key:       key [3]
##    key         index    moy
##    <chr>       <mth>  <dbl>
##  1 Series 1 1961 jan -0.626
##  2 Series 1 1961 fév  0.184
##  3 Series 1 1961 mar -0.836
##  4 Series 1 1961 avr  1.60 
##  5 Series 1 1961 mai  0.330
##  6 Series 1 1961 jui -0.820
##  7 Series 1 1961 jul  0.487
##  8 Series 1 1961 aoû  0.738
##  9 Series 1 1961 sep  0.576
## 10 Series 1 1961 oct -0.305
## # ... with 20 more rows
\end{verbatim}
\end{frame}

\end{document}
